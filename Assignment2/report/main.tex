\documentclass{article}

\usepackage[utf8]{inputenc}
\usepackage{graphicx} % For inserting images
\usepackage{minted}   % For typesetting code
\usepackage{eulervm}  % We suggest the Euler typeface for math, but feel free to 
\usepackage{charter}  % This typeface works well with Euler
\usepackage{xcolor}

\usepackage{amsmath}
\usepackage{amssymb}
\usepackage{float}
\usepackage{bm}

\usepackage[T1]{fontenc}       % Use T1 encoding

\title{Assignment X}

% Single author assignments
\author{FirstName LastName (Student number)}

% Group  assignments
% \author{FirstName1 LastName2 (Student Number) \\ FirstName2 LastName2 (Student number)\\ FirstName3 LastName3 (Student number)}

\date{October 2023}


% blackboard
\newcommand{\R}{{\mathbb R}}
\newcommand{\N}{{\mathbb N}}
\newcommand{\E}{{\mathbb E}}

% bold for vectors and matrices
\newcommand{\U}{{\mathbf U}}
\newcommand{\V}{{\mathbf V}}
\newcommand{\W}{{\mathbf W}}
\newcommand{\bu}{{\mathbf u}}
\newcommand{\bv}{{\mathbf v}}
\newcommand{\bw}{{\mathbf w}}
\newcommand{\x}{{\mathbf x}}
\newcommand{\y}{{\mathbf y}}
\newcommand{\bt}{{\mathbf t}}
\newcommand{\z}{{\mathbf z}}
\newcommand{\X}{{\mathbf X}}
\newcommand{\Y}{{\mathbf Y}}
\newcommand{\Z}{{\mathbf z}}

% gray
\definecolor{kc}{HTML}{999999}

% blue 1
\definecolor{cb}{HTML}{3F6797}
\definecolor{cbl}{HTML}{4F81BD}
\definecolor{cbll}{HTML}{7BA1CD}
\definecolor{cblll}{HTML}{A7C0DE}

% blue 2
\definecolor{cc}{HTML}{006EA6}
\definecolor{ccl}{HTML}{0089CF}
\definecolor{ccll}{HTML}{40A7DB}
\definecolor{cclll}{HTML}{7FC4E7}

% green
\definecolor{cg}{HTML}{7C9647}
\definecolor{cgl}{HTML}{9BBB59}
\definecolor{cgll}{HTML}{B4CC82}
\definecolor{cglll}{HTML}{CDDDAC}

% purple
\definecolor{cp}{HTML}{665082}
\definecolor{cpl}{HTML}{8064A2}
\definecolor{cpll}{HTML}{A08BB9}
\definecolor{cplll}{HTML}{BFB1D1}

% teal
\definecolor{ct}{HTML}{3C8A9E}
\definecolor{ctl}{HTML}{4BACC6}
\definecolor{ctll}{HTML}{78C1D4}
\definecolor{ctlll}{HTML}{A5D5E3}

% orange
\definecolor{co}{HTML}{C67838}
\definecolor{col}{HTML}{F79646}
\definecolor{coll}{HTML}{F9B074}
\definecolor{colll}{HTML}{FBCBA2}

\newcommand{\kc}[1]{{\textcolor{kc}{ #1 }}}

\newcommand{\cb}[1]{{\textcolor{cb}{ #1 }}}
\newcommand{\cbl}[1]{{\textcolor{cbl}{ #1 }}}
\newcommand{\cbll}[1]{{\textcolor{cbll}{ #1 }}}
\newcommand{\cblll}[1]{{\textcolor{cblll}{ #1 }}}

\newcommand{\cc}[1]{{\textcolor{cc}{ #1 }}}
\newcommand{\ccl}[1]{{\textcolor{ccl}{ #1 }}}
\newcommand{\ccll}[1]{{\textcolor{ccll}{ #1 }}}
\newcommand{\cclll}[1]{{\textcolor{cclll}{ #1 }}}

\newcommand{\cg}[1]{{\textcolor{cg}{ #1 }}}
\newcommand{\cgl}[1]{{\textcolor{cgl}{ #1 }}}
\newcommand{\cgll}[1]{{\textcolor{cgll}{ #1 }}}
\newcommand{\cglll}[1]{{\textcolor{cglll}{ #1 }}}

\newcommand{\cp}[1]{{\textcolor{cp}{ #1 }}}
\newcommand{\cpl}[1]{{\textcolor{cpl}{ #1 }}}
\newcommand{\cpll}[1]{{\textcolor{cpll}{ #1 }}}
\newcommand{\cplll}[1]{{\textcolor{cplll}{ #1 }}}

\newcommand{\ct}[1]{{\textcolor{ct}{ #1 }}}
\newcommand{\ctl}[1]{{\textcolor{ctl}{ #1 }}}
\newcommand{\ctll}[1]{{\textcolor{ctll}{ #1 }}}
\newcommand{\ctlll}[1]{{\textcolor{ctlll}{ #1 }}}

\newcommand{\co}[1]{{\textcolor{co}{ #1 }}}
\newcommand{\col}[1]{{\textcolor{col}{ #1 }}}
\newcommand{\coll}[1]{{\textcolor{coll}{ #1 }}}
\newcommand{\colll}[1]{{\textcolor{colll}{ #1 }}}

% partial symbol in gray
\newcommand{\kp}{{\kc{\partial}}}

\DeclareMathOperator*{\argmin}{arg\,min}
\DeclareMathOperator*{\argmax}{arg\,max} 

\begin{document}

\maketitle

\section{Answers}

\paragraph{Question D} To compute the limiting distribution we first need to construct the transition probability matrix \(P = P_{xy}\), with the transition probabilities given by ??? (Question b). This will be a \(400 \times 400\) matrix, where the rows indicate the intial state and the columns the final state. The probability distribution vector \(\vec{\pi}\) will be of size \(400\) and each component wlil correspond to one of the possible states of the system, so that \(\vec{\pi}^T = (\pi_{11} \hdots~ \pi_{2020})\). Each subscript represents the number of items left for each item type. The detailed construction of this matrix can be found in thte Appendix.
We do:
\[
\vec{\pi}_{t+1}^T = \vec{\pi}_{t}^T P 
\]
with an arbitrary \(\vec{\pi}_0\) until convergence. The result is \(\vec{\pi}_*\), which is the limiting distribution. The long run average costs can be computed now easily by multiplying \(\vec{\pi}_*\) by the  expected costs \(c(\x)\). The expected costs for each state \(\x\) are just the holding costs for that state plus an extra cost of 5 if either \(x_1\) or \(x_2\) is equal to one. The costs can also be stored in a vector \(\mathbf{c}\) of the same dimensions as \(\vec{\pi}_*\) so the long-run average costs are then \(\phi_* = \mathbf{c}\cdot \vec{\pi}_*\). We get a value of 10.172 for \(\phi_*\). 

\paragraph{Question E} Before defining the Poisson equation we need to introduce the value function for this problem, \(V_t(\x)\). This gives the total expected costs for a given state \(\x\) when there are \(t\) time steps left. As before, we can construct a vecot \(\mathbf{V}\) with the same dimensions as \(\vec{\pi}\) that stores, for a given \(t\), the total expected costs of every state, so that \(\mathbf{V} = (V_{11} \hdots~V_{2020})\). With this, the Poisson equation is:
\[
    \mathbf{V} + \vec{\phi} = \mathbf{c} + P\mathbf{V}
\]
Where \(\vec{\phi}\) is just a vector of the right dimensions wiht all its entries equal to \(\phi\). 
We solve the equation using value iteration, this is, we do \(\mathbf{V}_{t+1} = \mathbf{c} + P\mathbf{V}_t\) until \(\mathbf{V}_{t+1} - \mathbf{V}_t \leq \epsilon\) for \(\epsilon = 10^{-5}\). When convergence is reached, \(\mathbf{V}_* = \mathbf{V} - \min\{\mathbf{V}\}\) and \(\phi_* = \mathbf{c} + P\mathbf{V} - \mathbf{V}\). We get a value of 10.172 for \(\phi_*\).

\appendix
\section{Appendix}

\end{document}
